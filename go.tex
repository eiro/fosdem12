\documentclass{beamer}
% \usepackage[frenchb]{babel}
\usepackage[T1]{fontenc}
\usepackage{takahashi,listings,color,amsmath}
\usepackage{ulem} % sout pour barrer
\usepackage{wasysym} % \smiley \frownie
\usepackage[T1]{fontenc}
\lstset{ %
    basicstyle=\ttfamily\small,
    language=Perl,
}

\newcommand<>{\F}[1]{ \begin{frame} #1 \end{frame} }
\newcommand<>{\FT}[2]{ \begin{frame} \frametitle{#1} #2 \end{frame} }
\newcommand<>{\C}[1]{ \F{\lstinputlisting{#1}} }
\newcommand<>{\H}[1]{ \F{\lstinputlisting[language=haskell]{#1}} }
\newcommand<>{\T}[1]{ \takahashi{#1} }
\newcommand<>{\L}[1]{
   % \begin{itemize}[<+->]  #1
   \begin{itemize}  #1
   \end{itemize}
}
\newcommand<>{\PL}[1]{
   \begin{itemize}[<+->]  #1
   \end{itemize}
}
\newcommand<>{\SL}[2]{
    \F{
        \frametitle{#1}
        \L{#2}
    }
}

\newcommand<>{\A}[2]{
    \begin{tabular}{#1} #2
    \end{tabular}
}

\newcommand<>{\TL}[2]{ \T{\A{#1}{#2}} }
\newcommand<>{\TC}[1]{ \T{\A{c}{#1}} }

\begin{document}
\F{
   \A{ll}{
    twitter  & @marcchantreux \\
    freenode & eiro \\
    homepage & http://khatar.phear.org \\
    github   & http://github.com/eiro \\
    % homepage & \url{http://khatar.phear.org} \\
    % github   & \url{http://github.com/eiro} \\
   }
}
\TC{ Perlude \\ FOSDEM 2012 }
\TC{"time's up talk" \\ \pause ask your questions}
\T{40mn}
\TC{ % \heartsuit \\
    love | \\
    \pause or GTFO \\
}
\T{remember fibonacci ...}
\C{fibo1.pl}
\T{dead simple}
\T{what if ...}
\T{5 first elements containing 3 ?}
\C{fibo_broken.pl}
\T{\frownie}
\SL{\frownie}{
        \item broken fibo behaviour
        \item have to deal with a counter
}
\TC{i use sh}
\C{simple.sh}
\TC{
    still dead simple \\
    no matter the size of input \\
}
\TC{pipe is "on demand" operator}
\TC{fibo is a generator \\ head and grep are filters}
\TC{also, composition is easy 
    \\ G | F => G
    \\ F | F => F
}
\T{On demand = generator + filters}
\T{ p3rl p0wn3d! }
\T{ I can haz | in perl ? }
\TC{ overloading | ? \\ \pause needs OO syntax }
\TC{ \sout{overloading |} }
\T{steal elsewhere}
\T{steal haskell}
\H{fibo.hs}
\H{fibo2.hs}
\TC{Haskell prelude\\ take, drop, dropWhile, takeWhile, fold, ... }
\T{I can haz prelude in perl ?}
\T{generators encapulate states }
\T{=> generators return closures }
\T{closures are variables}
\T{variables can be passed as functions}
\T{welcome to HOP}
\begin{frame}
    \frametitle{from shell command to closure}
    \begin{tabular}{l@{ => }l}
     @ARGV     & @\_    \\
     while (1) & sub   \\
     say       & return (optionnal) \\
    \end{tabular}
\end{frame}
% \FT{from shell command to closure}{
%  \A{l@{ => }l}{
%      @ARGV     & @\_    \\
%      while (1) & sub   \\
%      say       & return (optionnal) \\
%   }
% }

% \C{fibo1.pl}
% \C{fibo_closure_p1.pl}
% \C{fibo_perlude.pl}
% \T{i can write}
% \C{fibo_closure_p2.pl}
% \TC{
%     G | F => G \\
%     and G is a closure \\
%     => F takes closure as argument
% }
% \C{fibo_closure_p2.pl}
% \TC{
%     F | F => F \\
%     => F returns a closure
% }
% \T{think about it ...}
% \T{i can write}
% \C{examples/takewhile.pl}
% \C{fibo1.pl}
% \TC{
%     easy \\
%     now use it \\
% }
% \C{fibo_closure_p2.pl}
% \TC{
%     counter still there \\
%     \frownie
% }
% \T{i want pipe \pause in perl}

\end{document}
